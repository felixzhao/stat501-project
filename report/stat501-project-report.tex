% Standard LaTeX document template
%  BE SURE TO PROCESS DOCUMENT TWICE IF IT CONTAINS CROSS-REFERENCES!

\documentclass[12pt]{article}
\usepackage[round]{natbib} %allow to set the bibliography style and
% import the bibliography file. See Bibliography management with
% natbib for more information on
% https://www.sharelatex.com/learn/Bibliography_management_with_natbib.
% See the reference sheet for natbib on
% http://merkel.zoneo.net/Latex/natbib.php.
% Several .bst files can be
% downloaded from http://kinglab.eeb.lsa.umich.edu/pub/biblios/bst/

\usepackage{graphicx,epsfig}
\usepackage{amssymb,amsmath,amsfonts,bm,color,supertabular,longtable,multirow}
\usepackage[colorlinks=true,linkcolor=black,citecolor=black,urlcolor=black]{hyperref}

\setlength{\oddsidemargin}{0in} % left margin, odd pages
\setlength{\evensidemargin}{0in} % left margin, even pages
\setlength{\textwidth}{6.5in} % widtth of text on page
\setlength{\topmargin}{-.3in} % add to default 1 in
\setlength{\headsep}{0in}     % add to default 25pt
\setlength{\textheight}{8.7in}  % height of text on page
\setlength{\parskip}{.1in}            % vertical space between paragraphs
\setcounter{tocdepth}{2}

%\setlength{\parindent}{0in}            % amount of indentation of paragraph


%  newcommands -- more newcommands used in the document.
%  not just in the preamble

\newcommand{\Var}{\mbox{Var}}
\newcommand{\Cov}{\mbox{Cov}}
\newcommand{\E}{\mbox{E}}
\newcommand{\ubeta}{\mbox{\boldmath$\beta$}}
% Independence symbol
\newcommand\independent{\protect\mathpalette{\protect\independenT}{\perp}}
\def\independenT#1#2{\mathrel{\rlap{$#1#2$}\mkern2mu{#1#2}}}


\title{STAT 501 Case Study Assignment} 
\author{Quan Zhao\\
School of Mathematics and Statistics\\ Victoria University of Wellington, New Zealand} 
%\date{}  % Add \date{} to make a blank date.


%  main body of document

\begin{document}

% Titlepage
\maketitle

\begin{abstract}
  This course provides students with an opportunity to develop their
  research skills in Mathematics and Statistics, including use of
  library resources, constructing literature reviews, developing
  research questions, writing research proposals, and developing
  skills in oral presentation. The template file gives an introduction
  to LaTeX,
\end{abstract}


% Table of Contents
\tableofcontents


\setlength{\baselineskip}{0.25in} % min space from bottom of one line

                                 % to top of next in a paragraph

                                 % place after \begin{document}



\newpage  % start from a new page
\section{Introduction}

\label{s.intro}

Context and background information.
\cite{Liu05}

\section{Objectives and Scope}

Specific objectives of the statistical consultation.
Scope of the analysis including what is and is not covered.

% \section{Methodology}

% Overview of the statistical methods used.
% Software and tools used for analysis.

\section{Data Description}

Sources of data.
Description of variables.
Data collection methods.
Data quality and validation procedures.

% \subsection{Citizen variables (Micheal)}

\subsection{Feature construction}
		
		\subsubsection{Body mass index}
		
		The \cite{WHO_BMI} defines body mass index (BMI) as $\text{BMI} = \frac{\text{weight (kg)}}{\left[\text{height} (m)\right]^2}$. These indexes can then be further categorized into:
		
		\begin{itemize}
			\item $0 - 18.5$: Underweight
			\item $18.5-24.9$: Normal
			\item $25.0-29.9$: Pre-obese
			\item $30.0-34.9$: Obesity 1
			\item $35.0-39.9$: Obesity 2
			\item $40+$: Obesity 3
		\end{itemize}
		
		which were then assigned to each individual.

% \subsection{Registretion (Felix)}

\subsection{Citizen Registration Analysis (Felix)}

\subsubsection{Start end registrations}
The \textit{start end registrations} table aggregates data for each registration type of citizens. 
Despite the table containing 1048575 rows, a significant majority (1047272 rows) have their 'Type' set as "nan", indicating that these citizens have not specified a registration Type.

Only 357 unique citizens have defined a registration type. 
Fig~\ref{fig:countoftypes}

The distribution of the count of citizens against the number of types selected by them is illustrated in Fig~\ref{fig:howmanytypes}.

\begin{figure}[h]
\centering
\includegraphics[width=0.7\linewidth]{images/reg_types.png}
\caption{Count of types}
\label{fig:countoftypes}
\end{figure}

\begin{figure}[h]
\centering
\includegraphics[width=0.7\linewidth]{images/start_end_reg_count.png}
\caption{How many citizens select how many types in start\_end\_registrations}
\label{fig:howmanytypes}
\end{figure}

\subsubsection{Registrations per Goal}
The \textit{registrations per goal} table aggregates registration data by the goal of each registration type for citizens. 

Notably, a single registration type can be associated with multiple goals. There are 362 unique citizens with registration types in both the \textit{registrations per goal} and \textit{goal and registration} tables, differing from the distribution in \textit{start\_end\_registrations}.

\subsubsection{Goal and Registration}
The \textit{goal and registration} table provides detailed registration data for each type and its associated goals for citizens.

\subsubsection{Missing Citizens in Start end registrations}
Upon analysis, it was discovered that four citizens, namely 831841, 909811, 955121, and 1022212, were absent in the \textit{start end registration} table but present in both the \textit{registrations per goal} and \textit{goal and registration} tables.

\subsubsection{Table Selection for Aggregation at the Citizen Level}
The \textit{Start end registrations} table appears to be the most suitable for representing citizen registration activities since it is already aggregated at the registration type level, although, it lacks information on four citizens and excludes the "EXERCISE" category. because of, the \textit{registrations per goal} table detailed goal-wise data for each registration type might result in a substantial aggregation workload. The \textit{goal and registration} table's data is deemed too intricate and is thus not considered for this phase.

For a more comprehensive representation of citizen registration activities, 
We aggregated in registration type level for each citizen based on "valueCount"  
in the \textit{Start end registrations} table.

% \subsection{Text messages (Felix)}

\subsection{Text Messages Analysis (Felix)}

We have consolidated the text message data for individual citizens and performed an aggregation based on four primary characteristics: intervention by the citizen in the message, the auto-generation of the message, the presence of a video in the message, and the length of the message text.

\subsubsection{Features Description}

\begin{enumerate}
    \item \textbf{count\_intervention}: This feature represents the number of times a citizen intervened in the message.
    
    \item \textbf{count\_autogenerated}: This denotes the number of messages that were auto-generated.
    
    \item \textbf{count\_withvideo}: This feature indicates the number of messages that contain a video.
    
    \item \textbf{min\_messagetext\_length}: This represents the minimum length of the message text among all the messages for a given citizen.
    
    \item \textbf{avg\_messagetext\_length}: This provides the average length of the message text for a given citizen.
    
    \item \textbf{max\_messagetext\_length}: This denotes the maximum length of the message text among all the messages for a given citizen.
\end{enumerate}

These features provide a comprehensive overview of the text messaging behavior and content of individual citizens. By analyzing these features, we can gain insights into the communication patterns, preferences, and tendencies of the citizens.


\section{Preliminary Data Analysis}

Descriptive statistics.
Data visualization.
Identification of outliers or anomalies.

% \subsection{Missing data handle (Micheal)}

\subsection{Missing values and MICE}
		
		Missing values are often a problem when working with datasets. Some techniques to handle missing values include ignoring them, or imputing them. To impute means to replace these cells with missing values with another value. The problem now is to determine the most appropriate value to give the cell or, in a more appropriate and principal phrasing, determine the most appropriate value to give the \emph{individual}. This distinction is important as it gives emphasis on the need for context when imputing missing values.
		
		Single imputation encompasses several methods of imputation including mean imputation or regression imputation (fit $Y= \widehat\beta X + \epsilon$, where $Y$ is your missing variable and $X$ is the rest of your variables). \cite{jadhav1} says that it is assumed that the single imputation is the correct one precision is overstated, but there can never be absolute certainty about validity of imputed values. \cite{rubin1} developed a method for averaging the outcome across multiple imputed datasets which has been further developed into \textsf{R}'s package \textsf{mice}. Before we delve into \textsf{mice}, we first need to discuss what type of missingness our data is, and what does \emph{missingness} exactly mean?
		
		\subsubsection{Missingness}
		
		There are four main categories 

\section{Statistical Models and Techniques Used}

Statistical tests conducted.
Models fitted to the data.
Model validation techniques.

% \subsection{Clustering (Felix)}

\subsection{Citizen Clustering (Felix)}

our endeavor to cluster citizens based on their messaging behavior and associated features, we adopted the k-means clustering algorithm. To facilitate a more intuitive visualization of these clusters, we employed the t-SNE (t-distributed Stochastic Neighbor Embedding) technique. 

A salient observation from the clustering results is the emergence of a distinct cluster, differentiated from the others. A deeper analysis revealed that members of this cluster uniformly exhibited the characteristic of having the highest maximum message length. Specifically, the longest messages dispatched by these individuals are notably extensive in character count compared to their counterparts in other clusters.

Post a meticulous preprocessing and normalization phase of the features, and subsequent iterative experimentation with diverse cluster counts, a consensus was reached that a quintet of clusters—five in total—offered the most balanced distribution. This determination was substantiated by the insights gleaned 
from the T-SNE plot Figure~\ref{fig:cluster}.

\begin{figure}[h]
  \centering
  \includegraphics[width=0.7\linewidth]{images/Kmeans_5_clusters}
  \caption{Count of types}
  \label{fig:cluster}
  \end{figure}

% \subsection{PCA (Felix) }

\subsection{PCA Analysis (Felix)}

Principal Component Analysis (PCA) is a statistical procedure that orthogonally transforms the original variables of a dataset into a set of linearly uncorrelated variables known as Principal Components (PCs). These PCs are ordered such that the first few retain most of the variation present in all of the original variables. PCA is frequently employed for dimensionality reduction, especially in contexts where data variables are highly correlated.

In the context of our study, PCA was not just a tool for dimensionality reduction but was instrumental in discerning the key features characterizing each cluster. Our methodology involved executing PCA individually for each cluster, aiming to identify the components that cumulatively explained 95\% of the variance.

Subsequent to the PCA, we computed the cumulative loadings for all features across the identified principal components. Here, 'cumulative loading' refers to the aggregated contribution of each feature across the components, providing an indication of the significance of each feature in explaining the variance within the cluster.

To elucidate the relative importance of the original features, they were ranked based on their cumulative loadings. This ranking facilitated the identification of features that were most explanatory for each cluster.

Visually, our findings are presented in two distinct plots. 
Figure~\ref{fig:all_features} a radar plot provides a comprehensive view, showcasing the impact of all features for each cluster. 
In contrast, Figure~\ref{fig:top5_features} focuses on the top 5 impactful features for each cluster, offering a more concentrated perspective. This latter plot is particularly illuminating as it accentuates the distinguishing features between clusters, providing clear demarcations in their characteristics.

\begin{figure}[h]
  \centering
  \includegraphics[width=0.7\linewidth]{images/all_feature}
  \caption{impact of all features for each cluster}
  \label{fig:all_features}
  \end{figure}

  \begin{figure}[h]
    \centering
    \includegraphics[width=0.7\linewidth]{images/top5_features.png}
    \caption{impact of top 5 features for each cluster}
    \label{fig:top5_features}
    \end{figure}

\subsection{Survival analysis (Micheal)}

\section{Key Findings}

Results of the statistical analysis.
Interpretation of these results in context.

\subsection{TOP 5 impact feature for each clusters (Felix)}

Figure~\ref{fig:table} shows detial of top 5 impact features for each clusters.

A meticulous examination of the top features for each cluster reveals intriguing patterns and insights. Cluster 3 is particularly noteworthy. Citizens encompassed within this cluster tend to remain in the program for extended durations, receiving messages with higher frequency and greater length. A salient feature influencing this cluster is the registration type labeled 'Alcohol'. This prominence of 'Alcohol' as a defining feature warrants a deeper exploration to understand its implications and potential underlying causes.

Conversely, Cluster 4 exhibits a contrasting behavior. Citizens here often exit the program prematurely. Influential factors for this cluster encompass attributes like BMI, DIET, and MEDICINE. A point of particular concern is the prominence of individuals labeled as 'BMI underweight'. Such individuals might necessitate specialized attention and interventions to ensure their well-being and sustained participation.

It's imperative to understand the nuances of feature loadings. All the top features identified are in a positive direction. A feature's high loading across all principal components, especially in the context of Cluster 2, signifies its pivotal role in determining the direction of the principal components for the data points within that cluster. However, it's crucial to differentiate between the importance and magnitude of a feature. While the loading elucidates the significance of a feature in capturing the variance or structure of the data, it doesn't provide insights into its magnitude. Thus, even if a feature has the highest loading on the principal components, it doesn't inherently imply that its average or cumulative values will be the most pronounced across all clusters.

\begin{figure}[h]
  \centering
  \includegraphics[width=0.7\linewidth]{images/table.png}
  \caption{table: impact of top 5 features for each cluster}
  \label{fig:table}
  \end{figure}


% \subsection{Survival analysis and days in the program (Micheal)}

\section{Survival analysis}
		
		After clustering, some measure of determining the performance of each cluster with respect to how long the clusters remained in the program is desired. Survival analysis is perfect for this, where survival is measured by how long each citizen stays in the program. Given the data, survival probabilities up to a time $t$ (days) can be estimated using a Kaplan-Meier estimator. This survival estimator takes the form:
		
		$$
		\widehat{S}(t) = \prod_{i: t_i \leq t} \left( 1 - \frac{d_i}{n_i} \right)
		$$
		
		where $t_i$ is a day when an individual was censored or left the program, $d_i$ is the number of people who left the program at $t_i$, and $n_i$ is the number of people still in the program.
		
		
		
		\subsection{Censorship}
		
		The Kaplan-Meier estimator has the ability to account for events that are invalid. For example: an individual leaves the program due to an outside reason and not of their own accord (which is the outcome of interest: voluntarily leaving the program). An individual could be asked why they left the program, or the reason could be given by the coach, or there could simply be no reason why an individual left. These responses (or lack thereof) were recorded and individuals were deemed uncensored/censored according to the reasons given. The following examples give a general criteria for censorship:
		\begin{itemize}
			\item An individual left the program and there is no reason given: censored.
			\item An individual left the program but did not communicate when exactly they left: censored.
			\item An individual left the program and communicated when exactly they left: not censored.
			\item An individual completed the program: not censored.
		\end{itemize}
		
		\begin{figure}[h]
			\begin{center}
				\includegraphics[width=150mm]{images/censored, birth.png}
				\caption{Kaplan-Meier curve of survival probabilities per cluster}
				\label{fig:cens_surv}
			\end{center}
		\end{figure}

\section{Recommendations}

Suggested actions or decisions based on the findings.

\section{Limitations and Future Research}

Limitations in data or methodology.
Recommendations for future research.

\section{Conclusions}

Final summary and conclusions drawn from the statistical analysis.

% end of structure

\newpage
%
%\bibliographystyle{harvard}
\bibliographystyle{apalike3}
%\bibliographystyle{abbrv} %this is the same as plainnat but with last name fist 
%\bibliographystyle{unsrtnat}  % Sets the bibliography style
% unsrtnat. See the article about
% bibliography styles for more
% information on
% https://www.sharelatex.com/learn/Natbib_bibliography_styles
\setlength{\bibhang}{0pt}

\raggedright
\bibliography{bib_501report}


\end{document}
